% OWN
\usepackage{csvsimple}
\usepackage{multirow}
\usepackage{xcolor}
\usepackage{colortbl}
\usepackage{tabu}
\usepackage[utf8]{inputenc}
\usepackage[T1]{fontenc}

% ORIGINAL
\usepackage{rotating}% for sideways figures and tables
\usepackage{longtable}% for long tables

\usepackage{booktabs}
\usepackage[load-configurations=version-1]{siunitx} % newer version

 % The following command is just for this sample document:
\newcommand{\cs}[1]{\texttt{\char`\\#1}}

 % Define an unnumbered theorem just for this sample document:
\theorembodyfont{\upshape}
\theoremheaderfont{\scshape}
\theorempostheader{:}
\theoremsep{\newline}
\newtheorem*{note}{Note}

\usepackage[utf8]{inputenc}

 % change the arguments, as appropriate, in the following:
\jmlrvolume{1}
\jmlryear{2010}
\jmlrworkshop{Workshop Title}


 % Two authors with the same address
\author{\Name{Paweł Ksieniewicz}
	\Email{pawel.ksieniewicz@pwr.edu.pl}\\
	\addr 
	Department of Systems and Computer Networks\\
	Faculty of Electronics\\
	Wrocław University of Science and Technology
}

\editor{Editor's name}


\newcommand{\restable}[2] {

\scriptsize
\centering
\setlength{\tabcolsep}{3.5pt}
\def\arraystretch{1.15}
\begin{tabular}{@{}|ccccc|ccccc||ccccc|ccccc||ccc|r|}\hline%


\multicolumn{10}{|c||}{\multirow{2}{*}{\bfseries Without oversampled set}} &
\multicolumn{10}{ c||}{\multirow{2}{*}{\bfseries With oversampled set}} &
\multirow{5}{*}{\rotatebox[origin=c]{-90}{\bfseries OS}} &
\multirow{5}{*}{\rotatebox[origin=c]{-90}{\bfseries US}} &
\multirow{5}{*}{\rotatebox[origin=c]{-90}{\bfseries Full}} &
\multirow{5}{*}{\rotatebox[origin=c]{-90}{\bfseries Dataset}}
	
\\
\multicolumn{10}{|c||}{}&
\multicolumn{10}{c||}{}&
\multicolumn{3}{c|}{}&
	
	 \\
	
	\multicolumn{5}{|c|}{\bfseries All members} &
	\multicolumn{5}{c||}{\bfseries Reduced members} &
	\multicolumn{5}{c|}{\bfseries All members} &
	\multicolumn{5}{c||}{\bfseries Reduced members} 
	& & & & \\
	
	\multirow{2}{*}{\rotatebox[origin=c]{-90}{\textsc{reg}}} &
	\multirow{2}{*}{\rotatebox[origin=c]{-90}{\textsc{wei}}} &
	\multirow{2}{*}{\rotatebox[origin=c]{-90}{\textsc{con}}} &
	\multirow{2}{*}{\rotatebox[origin=c]{-90}{\textsc{nor}}} &
	\multirow{2}{*}{\rotatebox[origin=c]{-90}{\textsc{nc}}} &

	\multirow{2}{*}{\rotatebox[origin=c]{-90}{\textsc{reg}}} &
	\multirow{2}{*}{\rotatebox[origin=c]{-90}{\textsc{wei}}} &
	\multirow{2}{*}{\rotatebox[origin=c]{-90}{\textsc{con}}} &
	\multirow{2}{*}{\rotatebox[origin=c]{-90}{\textsc{nor}}} &
	\multirow{2}{*}{\rotatebox[origin=c]{-90}{\textsc{nc}}} &

	\multirow{2}{*}{\rotatebox[origin=c]{-90}{\textsc{reg}}} &
	\multirow{2}{*}{\rotatebox[origin=c]{-90}{\textsc{wei}}} &
	\multirow{2}{*}{\rotatebox[origin=c]{-90}{\textsc{con}}} &
	\multirow{2}{*}{\rotatebox[origin=c]{-90}{\textsc{nor}}} &
	\multirow{2}{*}{\rotatebox[origin=c]{-90}{\textsc{nc}}} &

	\multirow{2}{*}{\rotatebox[origin=c]{-90}{\textsc{reg}}} &
	\multirow{2}{*}{\rotatebox[origin=c]{-90}{\textsc{wei}}} &
	\multirow{2}{*}{\rotatebox[origin=c]{-90}{\textsc{con}}} &
	\multirow{2}{*}{\rotatebox[origin=c]{-90}{\textsc{nor}}} &
	\multirow{2}{*}{\rotatebox[origin=c]{-90}{\textsc{nc}}} 

	& & & &
	
	\\
	
	&&&&&&&&&&&&&&&&&&&&&&&
	\\\hline\hline
	
	\csvreader[head to column names,
	           late after line=\csvifoddrow{\\}{\\\rowcolor{gray!10!white}},
	           late after last line = \\\hline]
	{results/#1.csv}{}%
	{
	
	\ereg & \ewei & \ecwei & \enwei & \encwei &
	\eregr & \eweir & \ecweir & \enweir & \encweir &
	\eregos & \eweios & \ecweios & \enweios & \encweios & 
	\eregros & \eweiros & \ecweiros & \enweiros & \encweiros &
	\os & \us & \reg & 
	
	\multicolumn{1}{l|}{\emph{\dataset}}
	
	}%
\end{tabular}
\caption{Balanced accuracy scores obtained using #2 as a base classifier}

  
}